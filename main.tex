\documentclass{beamer}
\usepackage{mathspec}
\usepackage{xeCJK}
\setCJKmainfont{IPAPMincho}
\setCJKsansfont{IPAGothic}
\setCJKmonofont{IPAGothic}

\setmainfont{FreeSerif}
\setsansfont{FreeSans}
\setmonofont{Latin Modern Mono}

\usetheme{DarkConsole}

\setbeamertemplate{theorems}[normal font]

\title{\texttt{PPGCC} workshop}
\subtitle{Configurando ambiente Python com Virtualenv}
\author{Paulo Honório\footnote{\texttt{paulo.honorio@ppgcc.ifce.edu.br}}}

\begin{document}

\begin{frame}
  \maketitle
\end{frame}

\begin{frame}{Agenda}
  \tableofcontents
\end{frame}

\section{Contexto}
\begin{frame}{Contexto}
 \begin{itemize}
  \item リスト1\pause
  \item リスト2\pause
  \item リスト3
  \end{itemize}
\end{frame}

\section{Contexto}
\begin{frame}{Contexto}
 \begin{itemize}
  \item リスト1\pause
  \item リスト2\pause
  \item リスト3
  \end{itemize}
\end{frame}


\section{Contexto}
\begin{frame}{Contexto}
 \begin{itemize}
  \item リスト1\pause
  \item リスト2\pause
  \item リスト3
  \end{itemize}
\end{frame}
\section{Contexto}
\begin{frame}{Contexto}
 \begin{itemize}
  \item リスト1\pause
  \item リスト2\pause
  \item リスト3
  \end{itemize}
\end{frame}
\section{Contexto}
\begin{frame}{Contexto}
 \begin{itemize}
  \item リスト1\pause
  \item リスト2\pause
  \item リスト3
  \end{itemize}
\end{frame}



\section{Instalação}

\begin{frame}{b1}
  c1

  \pause

  \begin{enumerate}
  \item d1\pause
  \item d2\pause
  \item d3
  \end{enumerate}

  \pause

  \begin{itemize}
  \item リスト1\pause
  \item リスト2\pause
  \item リスト3
  \end{itemize}
\end{frame}

\begin{frame}{テスト2}
  数式のテスト.

  \begin{theorem}[Gauss積分]
    以下の等式が成り立つ:
    \begin{equation}
      \int_{-\infty}^\infty \mathrm{e}^{-x^2}\,\mathrm{d}x=\sqrt{\pi}.
    \end{equation}
  \end{theorem}
\end{frame}


\section{Configuração}

\begin{frame}{b1}
  c1

  \pause

  \begin{enumerate}
  \item d1\pause
  \item d2\pause
  \item d3
  \end{enumerate}

  \pause

  \begin{itemize}
  \item リスト1\pause
  \item リスト2\pause
  \item リスト3
  \end{itemize}
\end{frame}

\begin{frame}{テスト2}
  数式のテスト.

  \begin{theorem}[Gauss積分]
    以下の等式が成り立つ:
    \begin{equation}
      \int_{-\infty}^\infty \mathrm{e}^{-x^2}\,\mathrm{d}x=\sqrt{\pi}.
    \end{equation}
  \end{theorem}
\end{frame}

\section{Versionamento}
\begin{frame}
あああああああああああああああああああああああああああああああああああああああああああああああああああああああああああああああああああああああああああああああああああああああああああああああああああああああああああああああああああああああああああああああああああああああああああああああああああああああああああああああああああああああああああああああああああああああああああああああああああああああああああああああああああああああああああああああああああああああああああああああああああああああああああああああああああああああああああああああああああああああああああああああああああああああああああああああああああああああああああああああああああああああああああああああああああああああああああああああああああああああああああああああああああああああああああああああああああああああああああああああああああああああああああああああああああああああああああああああああああああああああああああああああああああああああああああああああああああああああああああああああああああああああああああああああああああああああああああああああああああああああああああああ
あああああああああああああああああああああああああああああああああああああああああああああああああああああああああああああああああああ
\end{frame}
\end{document}
\endinput
%%
%% End of file `example_DarkConsole.tex'.
